Brittle materials are prevalent in industrial applications. A fundamental engineering challenge is predicting where failure can occur in such materials. This involves determining where fracture nucleation will occur, as well as how these fractures propagate.

Researchers at Sandia National Laboratories have successfully used molecular dynamics (MD) simulations to model the effects leading to fracture nucleation and growth in silica-based glasses, given an atomic-level description of the material. However, these simulations are too computationally intensive to allow a systematic study of the relation between atomic structure and fracture behavior.

This Clinic project develops a novel approach to predicting fracture growth in silicate glasses, using a graph theoretic description of the material and machine learning. By training a supervised learning algorithm on MD simulation data, we aim to develop a surrogate model that rapidly and accurately predicts where fractures will emerge under multiple environmental conditions. The model will provide new insight into how local structure, beyond a simple ``weakest link'' description, leads to fracture and failure.
